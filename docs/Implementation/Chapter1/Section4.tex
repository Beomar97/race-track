\section{Competition Analysis}
	Research has shown that the game \textit{Racetrack}, also known as \textit{Vector Race}, has never been realized as a computer desktop game [1] [2]. Currently, only two mobile applications on \gls{Android} with the same concept as RaceTrack are being distributed on the \gls{Google Play Store} [3]. However, these games are not suitable for local multiplayer or educational purposes, as the screen size is limited by the smartphone.
	\\~\\
	Thanks to the implementation with the programming language \gls{Java}, RaceTrack is platform-independent. This gives the user the advantage of running RaceTrack on all different kinds of operating systems like Windows, macOS, or Linux. Besides, a larger audience is reached, and therewith a larger potential of paying customers.
	\\~\\
	RaceTrack distinguishes itself by adding the ability to load in custom tracks into the game and the option to play the game with distinct game modes, with each of them handling situations (e.g. when crashing in the game) differently. It is also planned to be able to add special items onto the track, e.g. boosts or obstacles. This feature can be activated during the creation of the game round.