\section{Random java listing}
This is just an inline example:
	\begin{lstlisting}[style=java,caption={A funny little listing example}]
/**
 * I also write docs ;)
 */
public class HelloWorld {
	private int mysecretnumber;

	public HelloWorld(int secret) {
		mysecretnumber = secret;
		sayHello();
	}

	private void sayHello() {
		System.out.println("Hello World")
	}

	// Start
	public static void main(String args[]){
		HelloWorld hw = new HelloWorld(Integer.parseInt(args[0]));
	}
}
	\end{lstlisting}

Now we have a look at a masterpice of coding:
\lstinputlisting[style=java,caption={A nice piece of code}]{../../src/main/java/com/pathfinder/racetrack/controller/OptionsMenuController.java}
\subsection{Picutre}


\begin{figure}[H]
	\centering
	\includegraphics[width=10cm,height=10cm,keepaspectratio,center]{img/RaceTrack_Logo.png}
	\caption{$\alpha = 2.000, \eta = 1.000$, RK4 and some blabla}
\end{figure}

%\gls{ }
%To print the term, lowercase. For example, \gls{maths} prints mathematics when used.
%\Gls{ }
%The same as \gls but the first letter will be printed in uppercase. Example: \Gls{maths} prints Mathematics
%\glspl{ }
%The same as \gls but the term is put in its plural form. For instance, \glspl{formula} will write formulas in your final document.
%\Glspl{ }
%The same as \Gls but the term is put in its plural form. For example, \Glspl{formula} renders as Formulas.