%%%%%%%%%%%%%%%%%%%%%%%%%%%%%%%%%%%%%%%%%
% Academic Title Page
% LaTeX Template
% Version 2.0 (17/7/17)
%
% This template was downloaded from:
% http://www.LaTeXTemplates.com
%
% Original author:
% WikiBooks (LaTeX - Title Creation) with modifications by:
% Vel (vel@latextemplates.com)
%
% License:
% CC BY-NC-SA 3.0 (http://creativecommons.org/licenses/by-nc-sa/3.0/)
% 
% Instructions for using this template:
% This title page is capable of being compiled as is. This is not useful for 
% including it in another document. To do this, you have two options: 
%
% 1) Copy/paste everything between \begin{document} and \end{document} 
% starting at \begin{titlepage} and paste this into another LaTeX file where you 
% want your title page.
% OR
% 2) Remove everything outside the \begin{titlepage} and \end{titlepage}, rename
% this file and move it to the same directory as the LaTeX file you wish to add it to. 
% Then add \input{./<new filename>.tex} to your LaTeX file where you want your
% title page.
%
%%%%%%%%%%%%%%%%%%%%%%%%%%%%%%%%%%%%%%%%%

% Papier und Sprache
\documentclass[a4paper,10pt]{article}
\usepackage[top=1in, bottom=1.25in, left=0.75in, right=0.75in, a4paper]{geometry}
\usepackage[utf8]{inputenc}
\usepackage{ucs}
\usepackage[ngerman]{babel}

% Schriftart
\usepackage{cmbright}
\usepackage{float}

%Farben
\usepackage[table,dvipsnames]{xcolor}
\definecolor{lightgray}{gray}{0.9}

% Tablen
\usepackage{tabularx}
\usepackage{makecell}
\usepackage[export]{adjustbox}
\setlength\extrarowheight{5pt}
\usepackage{subcaption}

%Mathe Packete
\usepackage{mathtools}
\usepackage{amssymb}
\usepackage{amsmath}
\usepackage{amsfonts}

% Kopf- und Fusszeile
\usepackage[headsepline,footsepline]{scrpage2}
\pagestyle{scrheadings}
\clearscrheadfoot
\renewcommand*{\headfont}{\normalfont}
\ihead{PhIT}
\chead{Auftrag 1}
\ohead{M. Forster, M. Berweger}
\ifoot{26.10.2019}
\cfoot{Version 1.0}
\ofoot{Seite $|$ \pagemark}

% Package for Titlepage
\usepackage{mathpazo}

\usepackage{listings}
\definecolor{commentgreen}{rgb}{0,0.6,0}
\definecolor{halfgray}{gray}{0.5}
\definecolor{mymauve}{rgb}{0.58,0,0.82}
\lstset{ 
  breaklines=true,                 		% sets automatic line breaking
  commentstyle=\color{commentgreen},    % comment style
  extendedchars=true,             		% lets you use non-ASCII characters; for 8-bits encodings only, does not work with UTF-8
  frame=single,	                   		% adds a frame around the code
  keywordstyle=\color{blue},       		% keyword style
  numbers=left,                    		% where to put the line-numbers; possible values are (none, left, right)
  numberstyle=\tiny\color{halfgray}, % the style that is used for the line-numbers
  rulecolor=\color{halfgray},         % if not set, the frame-color may be changed on line-breaks within not-black text (e.g. comments (green here))
  stringstyle=\color{mymauve},     		% string literal style
  tabsize=1,	                   		% sets default tabsize to 2 spaces
  title=\lstname,                   	% show the filename of files included with \lstinputlisting; also try caption instead of title
  showstringspaces=false,
}
\lstdefinelanguage{bm}{
	alsoletter=/,
    morekeywords={METHOD, STARTTIME, STOPTIME, DT, INIT, d/dt, sqrt},
    sensitive=TRUE, % keywords are not case-sensitive
    morecomment=[s]{\{}{\}}, % s is for start and end delimiter
}

%----------------------------------------------------------------------------------------
%	PACKAGES AND OTHER DOCUMENT CONFIGURATIONS
%----------------------------------------------------------------------------------------

%\usepackage[T1]{fontenc} % Output font encoding for international characters
\begin{document}
%\usepackage{mathpazo} % Palatino font -> Moved zo main.tex
\begin{titlepage} % Suppresses displaying the page number on the title page and the subsequent page counts as page 1
	\newcommand{\HRule}{\rule{\linewidth}{0.2mm}} % Defines a new command for horizontal lines, change thickness here
	
	\center % Centre everything on the page
	
	%------------------------------------------------
	%	Headings
	%------------------------------------------------
	\textsc{\LARGE ZHAW Zürcher Hochschule für Angewandte Wissenschaften}\\[1.5cm] % Main heading such as the name of your university/college
	%\textsc{\Large PhIT}\\[0.5cm] % Major heading such as course name
	%\textsc{\large Auftrag 1}\\[0.5cm] % Minor heading such as course title
	
	%------------------------------------------------
	%	Title
	%------------------------------------------------
	\HRule\\[0.6cm]
	\includegraphics[scale=0.4]{img/RaceTrack_Logo.png}
	%{\huge\bfseries Simulation eines Zweikörperproblems}\\[0.4cm] % Title of your document
	\HRule\\[1.5cm]
	
	%------------------------------------------------
	%	Author(s)
	%------------------------------------------------
	{\large\textit{Autoren}}\\[0.6cm]
	Marco \textsc{Forster}\\ % Your name
	forstma1@students.zhaw.ch\\~\\
	Manuel \textsc{Berweger}\\ % Your name
	berweman@students.zhaw.ch\\~\\
	Marvin \textsc{Tseng}\\ % Your name
	tsengmar@students.zhaw.ch\\~\\
	Dan \textsc{Hochstrasser}\\ % Your name
	hochsdan@students.zhaw.ch\\~\\
	
	%------------------------------------------------
	%	Date
	%------------------------------------------------
	\vfill\vfill\vfill % Position the date 3/4 down the remaining page
	{\large05. März 2020} % Date, change the \today to a set date if you want to be precise
	
	%------------------------------------------------
	%	Logo
	%------------------------------------------------
	%\vfill\vfill
	%\includegraphics[width=0.2\textwidth]{placeholder.jpg}\\[1cm] % Include a department/university logo - this will require the graphicx package
	 
	%----------------------------------------------------------------------------------------
	
	\vfill % Push the date up 1/4 of the remaining page

\end{titlepage}
\end{document}